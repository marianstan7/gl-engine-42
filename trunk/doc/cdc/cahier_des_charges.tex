\documentclass [a4 paper,11pt]{article}
\usepackage [francais]{babel}
\usepackage [utf8]{inputenc}
\usepackage [T1]{fontenc}
\usepackage{textcomp}
\usepackage{vmargin}
\setmarginsrb{2.5cm}{0.5cm}{2.5cm}{1,5cm}{1,5cm}{0.5cm}{1.0cm}{2.0cm}


\title {Cahier des charges : glEngine}
\author {Gaël jochaud du plessix\\
Loick Michard}
\date {2012}

\begin{document}
\maketitle

\section{Rendu 3D}

Le rendu 3D est une des parties importamtes d'un bon moteur 3D. 
Celui-ci doit permettre la création de scène réaliste, avec un rendu graphique se rapprochant le plus possible de la réalité.
Pour cela il doit implémenter un ensemble de caractéristiques permettant un rendu rapide et conforme à la réalité.
De plus une des contraintes du moteur et de ne jamais descendre en dessous de 24 FPS.

\subsection{Rasterization}
La rasterization doit être éffectué en utilisant les fonctionalités de la dernières version d'openGL 4.2.

\subsection{Fenêtre d'affichage}
La librairie glEngine doit pouvoir ouvrir une fenêtre permettant d'afficher le rendu 3D du moteur.

\subsection{Effets graphiques}

Elle doit implémenter la plus grande partie des fonctionnalités ci-dessous.

\subsubsection*{Lumière ambiente}
Ajout d'une couleur constante à tous les objets, simulant une luminosité ambiante.
\subsubsection*{Lumière directionelle}
Gestion de l'éclairage de la scène par une ou plusieurs lumières directionelles.
\subsubsection*{Lumière ponctuelle}
Gestion de l'éclairage de la scène par une ou plusieurs lumières ponctuelles.
\subsubsection*{Lumière spéculaire}
Reflet de la lumière sur les objets donnant un effet de brillance. Cette couleur peut être différente de la couleur de la lumière pour plus de réalisme.
\subsubsection*{Ombres}
Tous les types de lumières doivent produire des ombres sur les objets. 
Pour les objets statiques elle ne doit pas être recalculée à chaque itération, contrairement aux objets dynamiques afin d'optimiser les performances.
\subsubsection*{Cubemap}
Il doit y voir la possibilité de définir un environnement via une texture. 
Il sera implémenté sous forme de cube map entourant la scène.
Elle devra se refléter et se réfracter dans chaque objet.
\subsubsection*{Occlusion ambiante}
On doit pouvoir activer l'occlusion ambiante pour obtenir des ombres diffuse.
Elle doit être calculée une seule fois et non à chaque rendu.
\subsubsection*{Bump mapping}
Grâce à une texture spéciale définissant le relief d'un objet, le moteur doit pouvoir donner un effet de déformation et de relief sur l'objet.
\subsubsection*{Réflection}
Dans un premier temps les objets doivent pouvoir refléter l'environnement.
Le moteur pourra plus tard implémenter la réflection de la scène complète dans l'objet.
\subsubsection*{Réfraction}
Dans un premier temps les objets doivent pouvoir réferacter l'environnement.
Le moteur pourra plus tard implémenter la réfraction de la scène complète dans l'objet.
\subsubsection*{Transparence}
Grâce à une propriété de transparence propre à chaque objet, le moteur doit donner la possibilité de voir la totalité du reste de la scène à travers un objet par transparence.

\section{Gestion de la scène}
Grâce au moteur, on doit pouvoir créer une scène complète comportant tous un ensemble d'éléments.
\subsection{Modèles}

\subsubsection*{Création de mesh}
L'utilisateur doit pouvoir définir un ensemble de point, accompagné d'un ensemble de normale qui formeront un mesh.
\subsubsection*{Importation}
Un modèle doit pouvoir être importé à partir d'un ou plusieurs formats standards (.obj, .3ds, .blend, ...).
Chaque format doit être géré au maximum afin de supporter toutes les fonctionnalités de celui-ci.
On doit ensuite pouvoir récupérer chaque objet composant un modèle séparemment.
\subsubsection*{Texture}
Chaque objet doit avoir la possibilité d'être texturé.
Chaque face doit pouvoir définir ses propres coordonnés dans la texture.
\subsubsection*{Hiérarchie}
Tous les objets doivent être organisé dans un arbre hiérarchique.
Ainsi quand on applique une modification d'un parent, elle doit être appliqué à tous ses descendants.

\subsection{Lumières}
On doit pouvoir ajouter tous type de lumière à la scène.

\subsection{Caméras}
La scène doit posséder une caméra courante, 

\subsubsection*{Caméra perspective}
\subsubsection*{Caméra orthogonale}
\subsubsection*{Cible caméra}

\subsection{Positionnement objets}

\subsubsection*{Position}
\subsubsection*{Rotation}
\subsubsection*{Échelle}

\subsection{Objets statiques et dynamiques}

\section{Traitement des métadonnées}

\subsection{Culling}
\subsection{Division spatiale}
\subsection{Physique}
\subsection{Modèles simplifiés}

\section{API}

\section{Démonstrations}


\end{document}
